\chapter{Introduction}

This chapter introduces the motivations and objectives that have driven this work and describes the thesis structure.

In particular, in the Section \ref{introduction:motivations} I explain the main reasons for which I worked to the project, 
the Section \ref{introduction:objective} deals with the goal of the thesis and finally, section \ref{introduction:structure} 
presents how the thesis is organized.


\section{Motivations}
\label{introduction:motivations}
I have always loved study and use the newest technologies and the blockchain is a perfect representation of what is a new 
great technology, with a huge potential. Another area that has always fascinated me is the data analysis, and specifically 
the capacity to extract complex and structured information starting from a simple dataset.
This project allowed me to join these two passions with an approach composed by both a theoretical and a practical part.

Over that personal motivations there are also important business and research aspects to consider: blockchain is an hot theme 
in the industry and academia, some governaments started to study its potential implications, huge companies are investing a 
lot on it, and only in last two years there have been 3.7 million Google search results for blockchain.

Moreover the study of the systems behaviour (in this case system composed from smart contract and dapps) can help to found 
bugs or vulnerabilities on the software. 

This work can also help to represent the logical behaviour of the system in a graphical manner, allowing also for non 
technicians to understand the overall cycle.


\section{Objective}
\label{introduction:objective}
The thesis has two main goals: the querying of the blockchain and the analysis of the extracted informations through Process 
Mining discovery algorithms. The first objective consist in the extraction of data from the blockchain and then in the 
filtering of these data based on some rules defined by the user. The second goal involve the application and comparison of 
different discovery algorithms in order to understand bindings between blockchain transactions. 

The first aim is preparatory for the second because the analysis is done starting from a dataset built querying the blockchain. 
The overall process try to reproduce the one used in data analysis: first a dataset is generated then these data are analyzed 
in order to infer information.


\section{Thesis structure}
\label{introduction:structure}
\begin{itemize}

   \item \textbf{Blockchain}.
   The first chapter is dedicated to the Blockchain: it contains a general introduction to the technology and its base
   concepts, a deepening on Ethereum (product used in the thesis) and all the tools that this environment exposes.
   
   \item \textbf{Process Mining}.
   The second chapter deals with process mining: it presents some basic concepts, its advantages and some of the techniques 
   used to work with it focusing mainly on the different algorithms for model discovery.

   \item \textbf{Case studies}.
   In the third chapter are showed 3 different case studies of the application of process discovery algorithms to the transactions 
   on Ethereum. The chapter contains details about the approach used, results obtained and some considerations on that.

   \item \textbf{Design and implementation}.
   The forth chapter is the more practical of the thesis, it describes the development of an application that allow both the 
   querying of data on Ethereum and the inference of a process model starting from a set of transactions. Here are explained 
   the choises done, the discovered problems and relative solutions as well as tools and technologies used.

   \item \textbf{Conclusions and future works}.
   Thesis ends with some remark about the work done and how it can be improved.
   
\end{itemize}